%% %%%%%%%%%%%%%%%%%%%%%%%%%%%%%%%%%%%%%%%%%%%%%%%%%%%%%%%%%
%% -- Muster für Beamer
%% --  Rom-Beamer.tex
%% --  Definitionen etc. in ./preamble/Beamer-defn.tex
%% --  Stand:  2024/03/10
%% --  Mustervorlage
%% %%%%%%%%%%%%%%%%%%%%%%%%%%%%%%%%%%%%%%%%%%%%%%%%%%%%%%%%%
\documentclass[%
	,ngerman
	,hyperref={%
		,bookmarks
		,colorlinks=true
		,citecolor=blue
		,linkcolor=blue
		,urlcolor=blue
			}
	,smaller
	]{beamer}

%% -- Beamer Definitionen 
%% --
\usepackage{./preamble/Rom-Beamer} 	% nicht alle üblichen Makros kann man nutzen

%% -- mit \begin{frame}[allowframebreaks] und dann \framebreak kann man ein
%% -- Thema über mehrere "Seiten" behandeln; siehe unten
%% --
\renewcommand{\insertcontinuationtext}{(Forts.)} % 

%% --
\begin{document}
%
\title{Titel Vortrag}
\author{Name}
\date{Datum Vortrag} 

%% -- Titelseite
%% --
\frame{\titlepage}

%% --
\begin{frame} 		
	\frametitle{Vorbemerkungen}
	
	\begin{enumerate}[(i)]
		\item 
		Wir betrachten nur \ldots
		
		\item
		\enquote{usw.}
		
		\item
		\ldots
		
	\end{enumerate}
\end{frame}
% --
\begin{frame}
	\frametitle{Der Hauptsatz}

\begin{theorem}\label{prop:hauptsatz}
	
Ist $ f $ eine stetige reellwertige Funktion auf dem Intervall $ \left[ 0,1 \right] $, so ist
%
\[
  	F( t ) = \int_{ 0 }^{ t } f(s) \ds
\]
%
differenzierbar auf diesem Intervall und $ F'(t) = f(t) $ für alle $ t \in \left[ 0,1 \right] $.
\end{theorem}

\end{frame}

%% --
\begin{frame}[allowframebreaks]
	\frametitle{Aufzählungen}
%
\begin{block}{Itemize}
	\begin{itemize}

		\item 
		Erstes item
		
		\item
		Unteraufzählung
		
		\begin{itemize}
		
		\item
		Unterpunkt
		
		\end{itemize}
		
		\item
		Zweites item
		
	\end{itemize}
%
\end{block}	

%%--
\framebreak
%%--

\begin{block}{Äquivalenz}
	\begin{enumerate}[(a)]
		\item 
		$ \norm{ T } = r(T) $.
		
		\item
		B
	\end{enumerate}
\end{block}

\begin{block}{Nummeriert}
	\begin{enumerate}[(i)]
		\item 
		Erstes $ \abs{a} $
		
		\item
		Zweites
	\end{enumerate}
\end{block}
%
\end{frame}

% --
\begin{frame}
	\frametitle{Theorem und Folgerung}
	
	\begin{theorem}
		Theorem
	\end{theorem}
	
	\begin{corollary}
		Korollar
	\end{corollary}
\end{frame}

% --
\end{document}

